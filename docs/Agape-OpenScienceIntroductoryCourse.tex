% Options for packages loaded elsewhere
\PassOptionsToPackage{unicode}{hyperref}
\PassOptionsToPackage{hyphens}{url}
%
\documentclass[
]{book}
\usepackage{amsmath,amssymb}
\usepackage{lmodern}
\usepackage{ifxetex,ifluatex}
\ifnum 0\ifxetex 1\fi\ifluatex 1\fi=0 % if pdftex
  \usepackage[T1]{fontenc}
  \usepackage[utf8]{inputenc}
  \usepackage{textcomp} % provide euro and other symbols
\else % if luatex or xetex
  \usepackage{unicode-math}
  \defaultfontfeatures{Scale=MatchLowercase}
  \defaultfontfeatures[\rmfamily]{Ligatures=TeX,Scale=1}
\fi
% Use upquote if available, for straight quotes in verbatim environments
\IfFileExists{upquote.sty}{\usepackage{upquote}}{}
\IfFileExists{microtype.sty}{% use microtype if available
  \usepackage[]{microtype}
  \UseMicrotypeSet[protrusion]{basicmath} % disable protrusion for tt fonts
}{}
\makeatletter
\@ifundefined{KOMAClassName}{% if non-KOMA class
  \IfFileExists{parskip.sty}{%
    \usepackage{parskip}
  }{% else
    \setlength{\parindent}{0pt}
    \setlength{\parskip}{6pt plus 2pt minus 1pt}}
}{% if KOMA class
  \KOMAoptions{parskip=half}}
\makeatother
\usepackage{xcolor}
\IfFileExists{xurl.sty}{\usepackage{xurl}}{} % add URL line breaks if available
\IfFileExists{bookmark.sty}{\usepackage{bookmark}}{\usepackage{hyperref}}
\hypersetup{
  pdftitle={AGAPE: An introductory course to open science for early career researchers},
  pdfauthor={An Opening Doors initiative},
  hidelinks,
  pdfcreator={LaTeX via pandoc}}
\urlstyle{same} % disable monospaced font for URLs
\usepackage{color}
\usepackage{fancyvrb}
\newcommand{\VerbBar}{|}
\newcommand{\VERB}{\Verb[commandchars=\\\{\}]}
\DefineVerbatimEnvironment{Highlighting}{Verbatim}{commandchars=\\\{\}}
% Add ',fontsize=\small' for more characters per line
\usepackage{framed}
\definecolor{shadecolor}{RGB}{248,248,248}
\newenvironment{Shaded}{\begin{snugshade}}{\end{snugshade}}
\newcommand{\AlertTok}[1]{\textcolor[rgb]{0.94,0.16,0.16}{#1}}
\newcommand{\AnnotationTok}[1]{\textcolor[rgb]{0.56,0.35,0.01}{\textbf{\textit{#1}}}}
\newcommand{\AttributeTok}[1]{\textcolor[rgb]{0.77,0.63,0.00}{#1}}
\newcommand{\BaseNTok}[1]{\textcolor[rgb]{0.00,0.00,0.81}{#1}}
\newcommand{\BuiltInTok}[1]{#1}
\newcommand{\CharTok}[1]{\textcolor[rgb]{0.31,0.60,0.02}{#1}}
\newcommand{\CommentTok}[1]{\textcolor[rgb]{0.56,0.35,0.01}{\textit{#1}}}
\newcommand{\CommentVarTok}[1]{\textcolor[rgb]{0.56,0.35,0.01}{\textbf{\textit{#1}}}}
\newcommand{\ConstantTok}[1]{\textcolor[rgb]{0.00,0.00,0.00}{#1}}
\newcommand{\ControlFlowTok}[1]{\textcolor[rgb]{0.13,0.29,0.53}{\textbf{#1}}}
\newcommand{\DataTypeTok}[1]{\textcolor[rgb]{0.13,0.29,0.53}{#1}}
\newcommand{\DecValTok}[1]{\textcolor[rgb]{0.00,0.00,0.81}{#1}}
\newcommand{\DocumentationTok}[1]{\textcolor[rgb]{0.56,0.35,0.01}{\textbf{\textit{#1}}}}
\newcommand{\ErrorTok}[1]{\textcolor[rgb]{0.64,0.00,0.00}{\textbf{#1}}}
\newcommand{\ExtensionTok}[1]{#1}
\newcommand{\FloatTok}[1]{\textcolor[rgb]{0.00,0.00,0.81}{#1}}
\newcommand{\FunctionTok}[1]{\textcolor[rgb]{0.00,0.00,0.00}{#1}}
\newcommand{\ImportTok}[1]{#1}
\newcommand{\InformationTok}[1]{\textcolor[rgb]{0.56,0.35,0.01}{\textbf{\textit{#1}}}}
\newcommand{\KeywordTok}[1]{\textcolor[rgb]{0.13,0.29,0.53}{\textbf{#1}}}
\newcommand{\NormalTok}[1]{#1}
\newcommand{\OperatorTok}[1]{\textcolor[rgb]{0.81,0.36,0.00}{\textbf{#1}}}
\newcommand{\OtherTok}[1]{\textcolor[rgb]{0.56,0.35,0.01}{#1}}
\newcommand{\PreprocessorTok}[1]{\textcolor[rgb]{0.56,0.35,0.01}{\textit{#1}}}
\newcommand{\RegionMarkerTok}[1]{#1}
\newcommand{\SpecialCharTok}[1]{\textcolor[rgb]{0.00,0.00,0.00}{#1}}
\newcommand{\SpecialStringTok}[1]{\textcolor[rgb]{0.31,0.60,0.02}{#1}}
\newcommand{\StringTok}[1]{\textcolor[rgb]{0.31,0.60,0.02}{#1}}
\newcommand{\VariableTok}[1]{\textcolor[rgb]{0.00,0.00,0.00}{#1}}
\newcommand{\VerbatimStringTok}[1]{\textcolor[rgb]{0.31,0.60,0.02}{#1}}
\newcommand{\WarningTok}[1]{\textcolor[rgb]{0.56,0.35,0.01}{\textbf{\textit{#1}}}}
\usepackage{longtable,booktabs,array}
\usepackage{calc} % for calculating minipage widths
% Correct order of tables after \paragraph or \subparagraph
\usepackage{etoolbox}
\makeatletter
\patchcmd\longtable{\par}{\if@noskipsec\mbox{}\fi\par}{}{}
\makeatother
% Allow footnotes in longtable head/foot
\IfFileExists{footnotehyper.sty}{\usepackage{footnotehyper}}{\usepackage{footnote}}
\makesavenoteenv{longtable}
\usepackage{graphicx}
\makeatletter
\def\maxwidth{\ifdim\Gin@nat@width>\linewidth\linewidth\else\Gin@nat@width\fi}
\def\maxheight{\ifdim\Gin@nat@height>\textheight\textheight\else\Gin@nat@height\fi}
\makeatother
% Scale images if necessary, so that they will not overflow the page
% margins by default, and it is still possible to overwrite the defaults
% using explicit options in \includegraphics[width, height, ...]{}
\setkeys{Gin}{width=\maxwidth,height=\maxheight,keepaspectratio}
% Set default figure placement to htbp
\makeatletter
\def\fps@figure{htbp}
\makeatother
\setlength{\emergencystretch}{3em} % prevent overfull lines
\providecommand{\tightlist}{%
  \setlength{\itemsep}{0pt}\setlength{\parskip}{0pt}}
\setcounter{secnumdepth}{5}
\usepackage{booktabs}
\ifluatex
  \usepackage{selnolig}  % disable illegal ligatures
\fi
\usepackage[]{natbib}
\bibliographystyle{apalike}

\title{AGAPE: An introductory course to open science for early career researchers}
\author{An Opening Doors initiative}
\date{}

\begin{document}
\maketitle

{
\setcounter{tocdepth}{1}
\tableofcontents
}
\includegraphics[width=0.6\textwidth,height=\textheight]{images/agapecover.png}

\hypertarget{introduction}{%
\chapter*{Introduction}\label{introduction}}
\addcontentsline{toc}{chapter}{Introduction}

Greetings, fellow PhD student or open science curious friend!

In this course, we would like to introduce you to the world of open science. Whether you are familiar with some of its concepts and resources or the open science movement doesn't ring any bells, we believe that what you learn here will be interesting for you and at the same time highly useful for your future career.

We ourselves are PhD students who firstly met during the course focusing on open and collaborative research. And because we felt that what we learned was very helpful and other students should have an opportunity to get familiar with these concepts too, we decided to create Agape. Agape means wide open, such as open science we want to promote. The word agapē originates from Greek and means love that is unconditional, such as our love for science. Under Agape we aim to disseminate open science between students, starting with this course and continuing with series of workshops where we can learn, exchange our opinions and experiences and together change the future.

With this course, Agape would like to open doors for you into the world of open science and to introduce various concepts that we think are very important but we were not told about. Whilst we all heard about the scientific integrity and open access publishing at some point of our studies, a domain of open science encompasses a much larger area. Given its extent, this course is by far not covering the whole scope of open science. However, during the course we provide you with useful links to other resources should you wish to learn more and start practising open science.

The course is structured into chapters that are written to expand on various topics. We think that an order they follow is logical and later chapters are building on knowledge in the previous ones, but you can decide to go through them in whatever order you like by clicking on different chapters in the menu on the left or to return to some of them should you find something is not clear or you forgot in the meantime.

Your progress throughout the course is tracked. At the end, you can obtain a Certificate of achievement. This will be generated for you automatically and emailed to the email address you use to log into this course. In order to obtain this certificate you'll need to achieve at least 90\% success rate in MCQs and activities at the end of each chapter. You have as many attempts to pass each of them as you want. Once you're happy with your result you confirm it and it will be saved under your profile until you confirm this for all chapters. After completing the last one, allow it a couple of hours before you receive the certificate.

Should you experience any technical problems or do not receive a certificate email us on

If you'd like to connect with us or see what's new you can do so on Twitter or Instagram (or any other social media account).

And now, without further delay, let's quench that thirst for knowledge!
\#\# Structure of the book \{-\}

\hypertarget{how-to-read-this-book}{%
\subsection*{How to read this book}\label{how-to-read-this-book}}
\addcontentsline{toc}{subsection}{How to read this book}

\hypertarget{how-to-read-the-web-edition}{%
\subsection*{How to read the web edition}\label{how-to-read-the-web-edition}}
\addcontentsline{toc}{subsection}{How to read the web edition}

Try these toolbar features located near the top of your browser:

\begin{itemize}
\tightlist
\item
  Menu
\item
  Search
\item
  Font to adjust text size and display
\item
  View source code on GitHub (if available)
\item
  Download book files (if available)
\item
  Shortcuts (arrow keys to navigate; \texttt{s} to toggle sidebar; \texttt{f} to toggle search)
\item
  Social Media
\item
  Share
\end{itemize}

\begin{figure}
\centering
\includegraphics{images/toolbarimage.png}
\caption{Toolbar features in open-access web edition}
\end{figure}

\hypertarget{what-did-we-leave-out}{%
\section*{What did we leave out?}\label{what-did-we-leave-out}}
\addcontentsline{toc}{section}{What did we leave out?}

\hypertarget{about-opening-doors-project}{%
\section*{About Opening Doors project}\label{about-opening-doors-project}}
\addcontentsline{toc}{section}{About Opening Doors project}

\hypertarget{meet-the-authors}{%
\section*{Meet the authors}\label{meet-the-authors}}
\addcontentsline{toc}{section}{Meet the authors}

\hypertarget{acknowledgement}{%
\section*{Acknowledgement}\label{acknowledgement}}
\addcontentsline{toc}{section}{Acknowledgement}

\hypertarget{how-to-contribute-to-this-project}{%
\section*{How to contribute to this project}\label{how-to-contribute-to-this-project}}
\addcontentsline{toc}{section}{How to contribute to this project}

\hypertarget{disclaimer}{%
\subsection*{Disclaimer}\label{disclaimer}}
\addcontentsline{toc}{subsection}{Disclaimer}

The information is this book is provided without warranty. The authors and publisher have neither liability nor responsibility to any person or entity related to any loss or damages arising from the information contained in this book.

In tip boxes like this one, we'll point out design tips, to help you keep your page looking looking

These tips highlight advice and tricks from community members.

As you work, you may start a local server to live preview this HTML book. This preview will update as you edit the book when you save individual .Rmd files. You can start the server in a work session by using the RStudio add-in ``Preview book'', or from the R console:

\begin{Shaded}
\begin{Highlighting}[]
\NormalTok{bookdown}\SpecialCharTok{::}\FunctionTok{serve\_book}\NormalTok{()}
\end{Highlighting}
\end{Shaded}

\hypertarget{introduction-to-open-science}{%
\chapter{Introduction to open science}\label{introduction-to-open-science}}

Bullet points:
Defining science: ``systematic enterprise that builds and organise knowledge in the form of testable explanations and predictions about the universe'' {[}\ldots{]} science performs knowledge validation ``through (the) sharing of findings and data and through peer review''. That is, sharing these knowledge bits with other scientists \emph{is} the method for evaluating their truthiness and validity. This process is called \emph{peer-reviewing}.
Isn't science open already? What is the current ``openness'' status and how it changed over time? When and where does the term open science come from?
Why we do NOT need open science:
The misuse of openly available dual-edge knowledge
The misinterpretation and misunderstanding of openly available knowledge from the amateurs and the general public.
The degradation of the peer-review process and the advent of predatory ``open'' journals.
The commodification of scientific knowledge, where knowledge is only apparently /free/ but the user becomes the product
The monopoly of the western world in designing open science.
The costs of higher accessibility to data and publications.
The absence of metrics in evaluating and recognizing one's efforts in generating and maintaining openly available knowledge.
The risks of extreme rigour hindering exploratory research.
Why we DO need open science:
The open science currents: opening what and how?
Developing and building infrastructure that help scientists practice open science
Implement new evaluation processes for scientists to progress in their career based also upon their degree of ``openness''
Dedicating time to improve knowledge divulgation, comprehensibility, and accessibility
Removing legal barriers that limit a complete access to the generated knowledge
Pragmatic approach (``as open as possible as close as necessary'')
The key challenges
Going past the traditional scientific community
Knowledge validation without economical barriers or conflict of interest
Multilingual knowledge
Educating a new generation of scientists
Costs and infrastructures
Monitoring the status of open science

\hypertarget{open-research-open-data-open-access}{%
\chapter{Open research, open data, open access}\label{open-research-open-data-open-access}}

\hypertarget{marks-quiz-module}{%
\section{Mark's quiz module}\label{marks-quiz-module}}

Loading\ldots{}

\hypertarget{pros-and-cons}{%
\chapter{Pros and cons}\label{pros-and-cons}}

\hypertarget{testing-google-form-based-quiz}{%
\section{Testing google form based quiz}\label{testing-google-form-based-quiz}}

Loading\ldots{}

\hypertarget{research-data-lifecycle}{%
\chapter{Research data lifecycle}\label{research-data-lifecycle}}

\hypertarget{create-quiz-using-learnr-package}{%
\section{Create quiz using learnr package}\label{create-quiz-using-learnr-package}}

\hypertarget{fair-principles}{%
\chapter{FAIR principles}\label{fair-principles}}

\hypertarget{test}{%
\section{test}\label{test}}

\hypertarget{data-centers-and-data-repositories}{%
\chapter{Data centers and data repositories}\label{data-centers-and-data-repositories}}

\hypertarget{policies}{%
\chapter{Policies}\label{policies}}

\hypertarget{test-child-function}{%
\section{Test Child function}\label{test-child-function}}

\hypertarget{firstpage}{%
\subsection{firstpage}\label{firstpage}}

\hypertarget{data-ethics}{%
\chapter{Data ethics}\label{data-ethics}}

\hypertarget{minimal-q}{%
\subsection{Minimal Q:}\label{minimal-q}}

any header + at least two options.

\begin{itemize}
\tightlist
\item[$\boxtimes$]
  Correct Option
\item[$\square$]
  Incorrect Options needed for Moodle.
\end{itemize}

\hypertarget{markdown---code-highlighting}{%
\subsection{Markdown -\textgreater{} Code Highlighting}\label{markdown---code-highlighting}}

\begin{Shaded}
\begin{Highlighting}[]
\KeywordTok{def}\NormalTok{ take\_default\_list(xs }\OperatorTok{=}\NormalTok{ []):}
\NormalTok{xs.append(}\DecValTok{7}\NormalTok{)}
\ControlFlowTok{return} \BuiltInTok{sum}\NormalTok{(xs) }\CommentTok{\# this is a comment.}
\end{Highlighting}
\end{Shaded}

What is the output of \texttt{take\_default\_list({[}1,2,3{]})}?

\begin{itemize}
\tightlist
\item[$\boxtimes$]
  It's impossible to know.
\item[$\square$]
  13
\item[$\square$]
  Something else.
\end{itemize}

\hypertarget{headings-trigger-new-questions}{%
\subsection{Headings trigger new questions!}\label{headings-trigger-new-questions}}

What is 3 + 4?

Ordered options never get shuffled!

\begin{enumerate}
\def\labelenumi{\arabic{enumi}.}
\tightlist
\item[$\square$]
  5
\item[$\square$]
  6
\item[$\boxtimes$]
  7
\item[$\square$]
  8
\end{enumerate}

\hypertarget{dictionary-review}{%
\section{Dictionary Review:}\label{dictionary-review}}

\begin{Shaded}
\begin{Highlighting}[]
\NormalTok{x }\OperatorTok{=}\NormalTok{ \{}\DecValTok{1}\NormalTok{: }\DecValTok{2}\NormalTok{, }\DecValTok{3}\NormalTok{: }\DecValTok{4}\NormalTok{, }\DecValTok{5}\NormalTok{: }\DecValTok{6}\NormalTok{\}}
\BuiltInTok{print}\NormalTok{(x[}\DecValTok{4}\NormalTok{])}
\end{Highlighting}
\end{Shaded}

What happens here?

\begin{itemize}
\tightlist
\item[$\square$]
  3
\item[$\square$]
  4
\item[$\square$]
  True
\item[$\square$]
  False
\item[$\boxtimes$]
  Crash!
\end{itemize}

\hypertarget{coding-and-other-skills}{%
\chapter{Coding and other skills}\label{coding-and-other-skills}}

\hypertarget{communication-and-ethics-of-open-science}{%
\chapter{Communication and ethics of open science}\label{communication-and-ethics-of-open-science}}

\hypertarget{opening-your-research}{%
\chapter{Opening your research}\label{opening-your-research}}

\hypertarget{conclusion}{%
\chapter{Conclusion}\label{conclusion}}

\hypertarget{how-to-contact-us}{%
\chapter{How to contact us}\label{how-to-contact-us}}

\hypertarget{data-centers-and-data-repositories-1}{%
\chapter{Data centers and data repositories}\label{data-centers-and-data-repositories-1}}

\hypertarget{conclusion-1}{%
\chapter{Conclusion}\label{conclusion-1}}

  \bibliography{book.bib,packages.bib}

\end{document}
